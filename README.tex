\documentclass[12pt]{article}
\usepackage[spanish]{babel}
\usepackage[T1]{fontenc}
\usepackage[utf8]{inputenc}
\usepackage{geometry}
\usepackage{hyperref}
\usepackage{longtable}
\usepackage{tikz}
\usepackage{listings}
\usepackage{xcolor}
\usepackage{booktabs}
\usepackage{float}

\geometry{a4paper, margin=2.5cm}

% Configuración para código
\lstset{
    language=C#,
    basicstyle=\ttfamily\footnotesize,
    keywordstyle=\color{blue},
    commentstyle=\color{green!60!black},
    stringstyle=\color{red},
    numbers=left,
    numberstyle=\tiny,
    frame=single,
    breaklines=true,
    captionpos=b
}

\title{Quixo - Documentación del Proyecto}
\author{Grupo 9}
\date{\today}

\begin{document}

\maketitle

\section*{Integrantes}
\begin{longtable}{|l|l|l|l|}
\hline
\textbf{Nombre} & \textbf{Carné} & \textbf{Usuario Git} & \textbf{Correo Git} \\
\hline
Juan Carlos Sánchez García & FI18012021 & jcsg21 & juanksanchez21@gmail.com \\
Ronal Jesús Delgado Vásquez & FI20016028 & Ronal1301 & rjdvw2001@gmail.com \\
Lineth Leiva Vargas & FI18009940 & Li-desing & lleiva10893@ufide.ac.cr \\
\hline
\end{longtable}

\section*{Frameworks y Herramientas Utilizadas}
\begin{itemize}
    \item \textbf{ASP.NET MVC 5} (.NET Framework 4.8.1) - Framework web para desarrollo de aplicaciones MVC
    \item \textbf{Entity Framework 6} - ORM (Object-Relational Mapping) para acceso a datos
    \item \textbf{SQL Server Express} - Motor de base de datos relacional para persistencia de datos
    \item \textbf{Bootstrap 3.3.7} - Framework CSS para diseño responsivo de interfaces
    \item \textbf{jQuery 3.7.0} - Biblioteca JavaScript para manipulación del DOM
    \item \textbf{Visual Studio 2022} - IDE principal para desarrollo .NET
    \item \textbf{NuGet} - Gestor de paquetes para .NET
    \item \textbf{Git} - Sistema de control de versiones
    \item \textbf{GitHub} - Plataforma de alojamiento de código
    \item \textbf{LaTeX} - Sistema de composición tipográfica para documentación
    \item \textbf{PlantUML} - Herramienta para creación de diagramas UML
\end{itemize}

\section*{Tipo de Aplicación}
MPA (Multi-Page Application) - Aplicación web tradicional con múltiples páginas HTML

\section*{Arquitectura}
MVC (Modelo-Vista-Controlador) con capas adicionales:
\begin{itemize}
    \item \textbf{Capa de Presentación}: Controllers, Views (Razor), Assets
    \item \textbf{Capa de Lógica de Negocio}: Reglas del juego, Servicios de aplicación
    \item \textbf{Capa de Acceso a Datos}: Entity Framework, DTOs, ViewModels
    \item \textbf{Capa de Datos}: Entidades del dominio, Base de datos relacional
\end{itemize}

\section*{Definición de la Base de Datos}

\subsection*{Motor de Base de Datos}
La aplicación utiliza \textbf{SQL Server Express} como motor de base de datos relacional. La conexión se configura en el archivo \texttt{Web.config}:

\begin{lstlisting}[caption=Configuración de conexión en Web.config]
<connectionStrings>
    <add name="QuixoConnection"
         connectionString="Data Source=(localdb)\MSSQLLocalDB;Initial Catalog=QuixoDB;Integrated Security=True;"
         providerName="System.Data.SqlClient" />
</connectionStrings>
\end{lstlisting}

\subsection*{Diagrama Entidad-Relación}

El diagrama entidad-relación completo de la base de datos Quixo está disponible en formato PNG en el repositorio de GitHub: \url{https://github.com/jcsg21/QuixoWebGrupo9/blob/main/docs/diagrams/modelo_datos.png}

\subsection*{Descripción Detallada de Tablas}

\subsubsection*{Tabla: Jugador}
Almacena información de los jugadores del sistema.
\begin{itemize}
    \item \textbf{Id}: Identificador único del jugador (autoincremental)
    \item \textbf{Nombre}: Nombre completo del jugador (máximo 100 caracteres)
    \item \textbf{Alias}: Apodo o nickname del jugador (máximo 50 caracteres)
    \item \textbf{PartidasJugadas}: Contador de partidas totales jugadas
    \item \textbf{PartidasGanadas}: Contador de partidas ganadas
\end{itemize}

\subsubsection*{Tabla: Equipo}
Gestiona los equipos para el modo de juego de 4 jugadores.
\begin{itemize}
    \item \textbf{Id}: Identificador único del equipo (autoincremental)
    \item \textbf{Nombre}: Nombre del equipo (máximo 100 caracteres)
    \item \textbf{PartidasJugadas}: Contador de partidas totales del equipo
    \item \textbf{PartidasGanadas}: Contador de partidas ganadas por el equipo
\end{itemize}

\subsubsection*{Tabla: Partida}
Entidad central que representa una partida de Quixo.
\begin{itemize}
    \item \textbf{Id}: Identificador único de la partida (autoincremental)
    \item \textbf{Modo}: Tipo de partida (1=Dos jugadores, 2=Cuatro jugadores)
    \item \textbf{FechaCreacion}: Fecha y hora de creación de la partida
    \item \textbf{DuracionTotal}: Tiempo total de duración de la partida
    \item \textbf{Resultado}: Estado final de la partida (EnCurso, GanoJugador1, etc.)
    \item \textbf{Jugador1Id - Jugador4Id}: Referencias a los jugadores participantes
    \item \textbf{EquipoAId, EquipoBId}: Referencias a los equipos (modo 4 jugadores)
\end{itemize}

\subsubsection*{Tabla: Movimiento}
Registra cada movimiento realizado durante una partida.
\begin{itemize}
    \item \textbf{Id}: Identificador único del movimiento (autoincremental)
    \item \textbf{NumeroMovimiento}: Número secuencial del movimiento en la partida
    \item \textbf{PartidaId}: Referencia a la partida correspondiente
    \item \textbf{JugadorId}: Jugador que realizó el movimiento
    \item \textbf{FilaOrigen, ColumnaOrigen}: Posición de origen del cubo movido
    \item \textbf{FilaDestino, ColumnaDestino}: Posición de destino del movimiento
    \item \textbf{DireccionEmpuje}: Dirección del empuje (Arriba, Abajo, Izquierda, Derecha)
    \item \textbf{FechaHora}: Timestamp del movimiento
\end{itemize}

\subsubsection*{Tabla: EstadoTablero}
Almacena snapshots del estado del tablero en cada movimiento.
\begin{itemize}
    \item \textbf{Id}: Identificador único del estado (autoincremental)
    \item \textbf{PartidaId}: Referencia a la partida correspondiente
    \item \textbf{NumeroMovimiento}: Número de movimiento correspondiente
    \item \textbf{TableroCompacto}: Representación serializada del tablero 5x5
    \item \textbf{SegundosTranscurridos}: Tiempo transcurrido hasta este estado
\end{itemize}

\subsection*{Script SQL de Creación de Tablas}

A continuación se muestra el script SQL que Entity Framework genera automáticamente para crear las tablas de la base de datos:

\begin{lstlisting}[language=SQL, caption=Script de creación de tablas]
-- Tabla Jugadores
CREATE TABLE [dbo].[Jugadors] (
    [Id] INT IDENTITY (1, 1) NOT NULL,
    [Nombre] NVARCHAR (100) NOT NULL,
    [Alias] NVARCHAR (50) NOT NULL,
    [PartidasJugadas] INT NOT NULL DEFAULT 0,
    [PartidasGanadas] INT NOT NULL DEFAULT 0,
    CONSTRAINT [PK_dbo.Jugadors] PRIMARY KEY CLUSTERED ([Id] ASC)
);

-- Tabla Equipos
CREATE TABLE [dbo].[Equipoes] (
    [Id] INT IDENTITY (1, 1) NOT NULL,
    [Nombre] NVARCHAR (100) NOT NULL,
    [PartidasJugadas] INT NOT NULL DEFAULT 0,
    [PartidasGanadas] INT NOT NULL DEFAULT 0,
    CONSTRAINT [PK_dbo.Equipoes] PRIMARY KEY CLUSTERED ([Id] ASC)
);

-- Tabla Partidas
CREATE TABLE [dbo].[Partidas] (
    [Id] INT IDENTITY (1, 1) NOT NULL,
    [Modo] INT NOT NULL,
    [FechaCreacion] DATETIME NOT NULL,
    [DuracionTotal] TIME (7) NOT NULL,
    [Resultado] INT NOT NULL,
    [Jugador1Id] INT NULL,
    [Jugador2Id] INT NULL,
    [Jugador3Id] INT NULL,
    [Jugador4Id] INT NULL,
    [EquipoAId] INT NULL,
    [EquipoBId] INT NULL,
    CONSTRAINT [PK_dbo.Partidas] PRIMARY KEY CLUSTERED ([Id] ASC)
);

-- Tabla Movimientos
CREATE TABLE [dbo].[Movimientoes] (
    [Id] INT IDENTITY (1, 1) NOT NULL,
    [NumeroMovimiento] INT NOT NULL,
    [PartidaId] INT NOT NULL,
    [JugadorId] INT NULL,
    [FilaOrigen] INT NOT NULL,
    [ColumnaOrigen] INT NOT NULL,
    [FilaDestino] INT NOT NULL,
    [ColumnaDestino] INT NOT NULL,
    [DireccionEmpuje] NVARCHAR (20) NOT NULL,
    [FechaHora] DATETIME NOT NULL,
    CONSTRAINT [PK_dbo.Movimientoes] PRIMARY KEY CLUSTERED ([Id] ASC)
);

-- Tabla EstadosTablero
CREATE TABLE [dbo].[EstadoTableroes] (
    [Id] INT IDENTITY (1, 1) NOT NULL,
    [PartidaId] INT NOT NULL,
    [NumeroMovimiento] INT NOT NULL,
    [TableroCompacto] NVARCHAR (50) NOT NULL,
    [SegundosTranscurridos] INT NOT NULL,
    CONSTRAINT [PK_dbo.EstadoTableroes] PRIMARY KEY CLUSTERED ([Id] ASC)
);

-- Índices y restricciones de clave foránea
CREATE NONCLUSTERED INDEX [IX_Jugador1Id] ON [dbo].[Partidas]([Jugador1Id] ASC);
CREATE NONCLUSTERED INDEX [IX_Jugador2Id] ON [dbo].[Partidas]([Jugador2Id] ASC);
CREATE NONCLUSTERED INDEX [IX_Jugador3Id] ON [dbo].[Partidas]([Jugador3Id] ASC);
CREATE NONCLUSTERED INDEX [IX_Jugador4Id] ON [dbo].[Partidas]([Jugador4Id] ASC);
CREATE NONCLUSTERED INDEX [IX_EquipoAId] ON [dbo].[Partidas]([EquipoAId] ASC);
CREATE NONCLUSTERED INDEX [IX_EquipoBId] ON [dbo].[Partidas]([EquipoBId] ASC);

ALTER TABLE [dbo].[Partidas] ADD CONSTRAINT [FK_dbo.Partidas_dbo.Jugadors_Jugador1Id]
    FOREIGN KEY ([Jugador1Id]) REFERENCES [dbo].[Jugadors] ([Id]);
ALTER TABLE [dbo].[Partidas] ADD CONSTRAINT [FK_dbo.Partidas_dbo.Jugadors_Jugador2Id]
    FOREIGN KEY ([Jugador2Id]) REFERENCES [dbo].[Jugadors] ([Id]);
ALTER TABLE [dbo].[Partidas] ADD CONSTRAINT [FK_dbo.Partidas_dbo.Jugadors_Jugador3Id]
    FOREIGN KEY ([Jugador3Id]) REFERENCES [dbo].[Jugadors] ([Id]);
ALTER TABLE [dbo].[Partidas] ADD CONSTRAINT [FK_dbo.Partidas_dbo.Jugadors_Jugador4Id]
    FOREIGN KEY ([Jugador4Id]) REFERENCES [dbo].[Jugadors] ([Id]);
ALTER TABLE [dbo].[Partidas] ADD CONSTRAINT [FK_dbo.Partidas_dbo.Equipoes_EquipoAId]
    FOREIGN KEY ([EquipoAId]) REFERENCES [dbo].[Equipoes] ([Id]);
ALTER TABLE [dbo].[Partidas] ADD CONSTRAINT [FK_dbo.Partidas_dbo.Equipoes_EquipoBId]
    FOREIGN KEY ([EquipoBId]) REFERENCES [dbo].[Equipoes] ([Id]);

CREATE NONCLUSTERED INDEX [IX_PartidaId] ON [dbo].[Movimientoes]([PartidaId] ASC);
CREATE NONCLUSTERED INDEX [IX_JugadorId] ON [dbo].[Movimientoes]([JugadorId] ASC);

ALTER TABLE [dbo].[Movimientoes] ADD CONSTRAINT [FK_dbo.Movimientoes_dbo.Partidas_PartidaId]
    FOREIGN KEY ([PartidaId]) REFERENCES [dbo].[Partidas] ([Id]) ON DELETE CASCADE;
ALTER TABLE [dbo].[Movimientoes] ADD CONSTRAINT [FK_dbo.Movimientoes_dbo.Jugadors_JugadorId]
    FOREIGN KEY ([JugadorId]) REFERENCES [dbo].[Jugadors] ([Id]);

CREATE NONCLUSTERED INDEX [IX_PartidaId] ON [dbo].[EstadoTableroes]([PartidaId] ASC);

ALTER TABLE [dbo].[EstadoTableroes] ADD CONSTRAINT [FK_dbo.EstadoTableroes_dbo.Partidas_PartidaId]
    FOREIGN KEY ([PartidaId]) REFERENCES [dbo].[Partidas] ([Id]) ON DELETE CASCADE;
\end{lstlisting}

\subsection*{Enumeraciones y Constantes}

\subsubsection*{ModoPartida}
\begin{itemize}
    \item \textbf{DosJugadores} = 1
    \item \textbf{CuatroJugadores} = 2
\end{itemize}

\subsubsection*{ResultadoPartida}
\begin{itemize}
    \item \textbf{EnCurso} = 0
    \item \textbf{GanoJugador1} = 1
    \item \textbf{GanoJugador2} = 2
    \item \textbf{GanoJugador3} = 3
    \item \textbf{GanoJugador4} = 4
    \item \textbf{GanoEquipoA} = 5
    \item \textbf{GanoEquipoB} = 6
\end{itemize}

\subsubsection*{CellOwner (para lógica del juego)}
\begin{itemize}
    \item \textbf{None} = 0 (Casilla vacía)
    \item \textbf{Player1} = 1 (Jugador 1/Equipo A)
    \item \textbf{Player2} = 2 (Jugador 2/Equipo B)
\end{itemize}

\subsubsection*{OrientationPoint (modo 4 jugadores)}
\begin{itemize}
    \item \textbf{None} = 0
    \item \textbf{Up} = 1
    \item \textbf{Right} = 2
    \item \textbf{Down} = 3
    \item \textbf{Left} = 4
\end{itemize}

\section*{Referencias y Recursos}

\subsection*{Sitios Web y Documentación}
\begin{itemize}
    \item \url{https://learn.microsoft.com/es-es/aspnet/mvc/overview/getting-started/introduction/getting-started} - Guía oficial de ASP.NET MVC
    \item \url{https://learn.microsoft.com/es-es/ef/ef6/} - Documentación de Entity Framework 6
    \item \url{https://learn.microsoft.com/es-es/sql/sql-server/?view=sql-server-ver16} - Documentación de SQL Server
    \item \url{https://getbootstrap.com/docs/3.3/} - Documentación de Bootstrap 3.3.7
    \item \url{https://jquery.com/} - Sitio oficial de jQuery
    \item \url{https://plantuml.com/} - Documentación de PlantUML para diagramas
    \item \url{https://github.com/} - Plataforma GitHub para control de versiones
    \item \url{https://q.agency/blog/asp-net-mvc-5-multi-level-convention-based-routing/} - Enrutamiento en ASP.NET MVC
    \item \url{https://www.entityframeworktutorial.net/code-first/what-is-code-first.aspx} - Documentación de Entity Framework 6
    \item \url{https://www.connectionstrings.com/sql-server/} - Connection Strings en SQL Express
    \item \url{https://www.c-sharpcorner.com/article/model-validation-in-mvc-5/} - ASP.NET MVC model validation attributes
    \item \url{https://stackoverflow.com/questions/38514621/entity-framework-tasks-and-async-await} - C\# async await patterns with Entity Framework
\end{itemize}

\subsection*{Uso de Agentes de IA}

\subsubsection*{GitHub Copilot}
Se utilizó GitHub Copilot durante el desarrollo para:
\begin{itemize}
    \item Generación de código repetitivo (constructores, propiedades)
    \item Sugerencias de métodos para manipulación de datos
    \item Ayuda en consultas LINQ para Entity Framework
    \item Sugerencias de validación y manejo de errores
\end{itemize}

\textbf{Prompts utilizados con Copilot:}
\begin{itemize}
    \item "Crear constructor para clase Partida con inicialización de colecciones"
    \item "Método para validar movimiento en tablero Quixo"
    \item "Consulta LINQ para obtener estadísticas de partidas por modo"
    \item "Configuración de rutas en ASP.NET MVC"
\end{itemize}

\subsubsection*{ChatGPT}
Se consultó ChatGPT para:
\begin{itemize}
    \item Resolución de problemas específicos de configuración
    \item Explicaciones de conceptos de Entity Framework
    \item Sugerencias de estructura de proyecto
    \item Ayuda con sintaxis de LaTeX para documentación
\end{itemize}

\textbf{Prompts utilizados con ChatGPT:}
\begin{itemize}
    \item "¿Cómo configurar Entity Framework 6 con SQL Server Express?"
    \item "Explicación de relaciones uno a muchos en EF Code First"
    \item "Cómo crear diagramas de base de datos en LaTeX con TikZ"
    \item "Sintaxis correcta para enumeraciones en C\# con Entity Framework"
\end{itemize}

\section*{Instructivo de Instalación y Ejecución}

\subsection*{Prerrequisitos}
\begin{itemize}
    \item Windows 10/11 (recomendado)
    \item Visual Studio 2022 con carga de trabajo ".NET desktop development"
    \item SQL Server Express 2019 o superior (incluido con Visual Studio)
    \item .NET Framework 4.8.1 (incluido con Visual Studio)
    \item Git para control de versiones
\end{itemize}

\subsection*{Instalación}
\begin{enumerate}
    \item \textbf{Clonar el repositorio:}
    \begin{lstlisting}[language=bash]
    git clone https://github.com/jcsg21/QuixoWebGrupo9.git
    cd QuixoWebGrupo9
    \end{lstlisting}

    \item \textbf{Abrir la solución en Visual Studio:}
    \begin{itemize}
        \item Ejecutar Visual Studio 2022
        \item Seleccionar "Abrir un proyecto o solución"
        \item Navegar a la carpeta del proyecto y seleccionar \texttt{PF/Quixo.Web/Quixo.Web.sln}
    \end{itemize}

    \item \textbf{Restaurar paquetes NuGet:}
    \begin{itemize}
        \item En Visual Studio: Tools $\rightarrow$ NuGet Package Manager $\rightarrow$ Restore NuGet Packages
        \item O desde la consola: Ejecutar \texttt{nuget restore Quixo.Web.sln}
    \end{itemize}

    \item \textbf{Verificar configuración de base de datos:}
    \begin{itemize}
        \item Abrir \texttt{Web.config}
        \item Verificar que la cadena de conexión \texttt{QuixoConnection} apunte a una instancia válida de SQL Server Express
        \item Por defecto usa \texttt{(localdb)\MSSQLLocalDB}
    \end{itemize}
\end{enumerate}

\subsection*{Compilación}
\begin{enumerate}
    \item En Visual Studio: Build $\rightarrow$ Build Solution (o Ctrl+Shift+B)
    \item Verificar que no existan errores de compilación
    \item La compilación generará los assemblies en la carpeta \texttt{bin/}
\end{enumerate}

\subsection*{Ejecución}
\begin{enumerate}
    \item \textbf{Ejecutar la aplicación:}
    \begin{itemize}
        \item Presionar F5 o Debug $\rightarrow$ Start Debugging
        \item La aplicación se ejecutará en IIS Express
        \item Se abrirá automáticamente el navegador predeterminado
    \end{itemize}

    \item \textbf{Creación automática de base de datos:}
    \begin{itemize}
        \item Entity Framework creará automáticamente la base de datos \texttt{QuixoDB} en SQL Server Express
        \item Las tablas se crearán según las entidades definidas en el código
        \item Se ejecutarán las migraciones iniciales si existen
    \end{itemize}

    \item \textbf{Verificación de funcionamiento:}
    \begin{itemize}
        \item La página principal debe cargar correctamente
        \item Probar crear una nueva partida
        \item Verificar que se puede navegar entre las diferentes secciones
    \end{itemize}
\end{enumerate}

\subsection*{Solución de Problemas Comunes}

\subsubsection*{Error de conexión a base de datos}
\begin{itemize}
    \item Verificar que SQL Server Express esté ejecutándose
    \item Comprobar que la instancia \texttt{(localdb)\MSSQLLocalDB} existe
    \item Alternativa: Cambiar la cadena de conexión a una instancia nombrada
\end{itemize}

\subsubsection*{Error de permisos}
\begin{itemize}
    \item Ejecutar Visual Studio como administrador
    \item Verificar permisos de escritura en la carpeta del proyecto
\end{itemize}

\subsubsection*{Paquetes NuGet no encontrados}
\begin{itemize}
    \item Limpiar caché de NuGet: \texttt{nuget locals all -clear}
    \item Restaurar paquetes nuevamente
\end{itemize}

\end{document}
